
% Default to the notebook output style

    


% Inherit from the specified cell style.




    
\documentclass[11pt]{article}

    
    
    \usepackage[T1]{fontenc}
    % Nicer default font (+ math font) than Computer Modern for most use cases
    \usepackage{mathpazo}

    % Basic figure setup, for now with no caption control since it's done
    % automatically by Pandoc (which extracts ![](path) syntax from Markdown).
    \usepackage{graphicx}
    % We will generate all images so they have a width \maxwidth. This means
    % that they will get their normal width if they fit onto the page, but
    % are scaled down if they would overflow the margins.
    \makeatletter
    \def\maxwidth{\ifdim\Gin@nat@width>\linewidth\linewidth
    \else\Gin@nat@width\fi}
    \makeatother
    \let\Oldincludegraphics\includegraphics
    % Set max figure width to be 80% of text width, for now hardcoded.
    \renewcommand{\includegraphics}[1]{\Oldincludegraphics[width=.8\maxwidth]{#1}}
    % Ensure that by default, figures have no caption (until we provide a
    % proper Figure object with a Caption API and a way to capture that
    % in the conversion process - todo).
    \usepackage{caption}
    \DeclareCaptionLabelFormat{nolabel}{}
    \captionsetup{labelformat=nolabel}

    \usepackage{adjustbox} % Used to constrain images to a maximum size 
    \usepackage{xcolor} % Allow colors to be defined
    \usepackage{enumerate} % Needed for markdown enumerations to work
    \usepackage{geometry} % Used to adjust the document margins
    \usepackage{amsmath} % Equations
    \usepackage{amssymb} % Equations
    \usepackage{textcomp} % defines textquotesingle
    % Hack from http://tex.stackexchange.com/a/47451/13684:
    \AtBeginDocument{%
        \def\PYZsq{\textquotesingle}% Upright quotes in Pygmentized code
    }
    \usepackage{upquote} % Upright quotes for verbatim code
    \usepackage{eurosym} % defines \euro
    \usepackage[mathletters]{ucs} % Extended unicode (utf-8) support
    \usepackage[utf8x]{inputenc} % Allow utf-8 characters in the tex document
    \usepackage{fancyvrb} % verbatim replacement that allows latex
    \usepackage{grffile} % extends the file name processing of package graphics 
                         % to support a larger range 
    % The hyperref package gives us a pdf with properly built
    % internal navigation ('pdf bookmarks' for the table of contents,
    % internal cross-reference links, web links for URLs, etc.)
    \usepackage{hyperref}
    \usepackage{longtable} % longtable support required by pandoc >1.10
    \usepackage{booktabs}  % table support for pandoc > 1.12.2
    \usepackage[inline]{enumitem} % IRkernel/repr support (it uses the enumerate* environment)
    \usepackage[normalem]{ulem} % ulem is needed to support strikethroughs (\sout)
                                % normalem makes italics be italics, not underlines
    

    
    
    % Colors for the hyperref package
    \definecolor{urlcolor}{rgb}{0,.145,.698}
    \definecolor{linkcolor}{rgb}{.71,0.21,0.01}
    \definecolor{citecolor}{rgb}{.12,.54,.11}

    % ANSI colors
    \definecolor{ansi-black}{HTML}{3E424D}
    \definecolor{ansi-black-intense}{HTML}{282C36}
    \definecolor{ansi-red}{HTML}{E75C58}
    \definecolor{ansi-red-intense}{HTML}{B22B31}
    \definecolor{ansi-green}{HTML}{00A250}
    \definecolor{ansi-green-intense}{HTML}{007427}
    \definecolor{ansi-yellow}{HTML}{DDB62B}
    \definecolor{ansi-yellow-intense}{HTML}{B27D12}
    \definecolor{ansi-blue}{HTML}{208FFB}
    \definecolor{ansi-blue-intense}{HTML}{0065CA}
    \definecolor{ansi-magenta}{HTML}{D160C4}
    \definecolor{ansi-magenta-intense}{HTML}{A03196}
    \definecolor{ansi-cyan}{HTML}{60C6C8}
    \definecolor{ansi-cyan-intense}{HTML}{258F8F}
    \definecolor{ansi-white}{HTML}{C5C1B4}
    \definecolor{ansi-white-intense}{HTML}{A1A6B2}

    % commands and environments needed by pandoc snippets
    % extracted from the output of `pandoc -s`
    \providecommand{\tightlist}{%
      \setlength{\itemsep}{0pt}\setlength{\parskip}{0pt}}
    \DefineVerbatimEnvironment{Highlighting}{Verbatim}{commandchars=\\\{\}}
    % Add ',fontsize=\small' for more characters per line
    \newenvironment{Shaded}{}{}
    \newcommand{\KeywordTok}[1]{\textcolor[rgb]{0.00,0.44,0.13}{\textbf{{#1}}}}
    \newcommand{\DataTypeTok}[1]{\textcolor[rgb]{0.56,0.13,0.00}{{#1}}}
    \newcommand{\DecValTok}[1]{\textcolor[rgb]{0.25,0.63,0.44}{{#1}}}
    \newcommand{\BaseNTok}[1]{\textcolor[rgb]{0.25,0.63,0.44}{{#1}}}
    \newcommand{\FloatTok}[1]{\textcolor[rgb]{0.25,0.63,0.44}{{#1}}}
    \newcommand{\CharTok}[1]{\textcolor[rgb]{0.25,0.44,0.63}{{#1}}}
    \newcommand{\StringTok}[1]{\textcolor[rgb]{0.25,0.44,0.63}{{#1}}}
    \newcommand{\CommentTok}[1]{\textcolor[rgb]{0.38,0.63,0.69}{\textit{{#1}}}}
    \newcommand{\OtherTok}[1]{\textcolor[rgb]{0.00,0.44,0.13}{{#1}}}
    \newcommand{\AlertTok}[1]{\textcolor[rgb]{1.00,0.00,0.00}{\textbf{{#1}}}}
    \newcommand{\FunctionTok}[1]{\textcolor[rgb]{0.02,0.16,0.49}{{#1}}}
    \newcommand{\RegionMarkerTok}[1]{{#1}}
    \newcommand{\ErrorTok}[1]{\textcolor[rgb]{1.00,0.00,0.00}{\textbf{{#1}}}}
    \newcommand{\NormalTok}[1]{{#1}}
    
    % Additional commands for more recent versions of Pandoc
    \newcommand{\ConstantTok}[1]{\textcolor[rgb]{0.53,0.00,0.00}{{#1}}}
    \newcommand{\SpecialCharTok}[1]{\textcolor[rgb]{0.25,0.44,0.63}{{#1}}}
    \newcommand{\VerbatimStringTok}[1]{\textcolor[rgb]{0.25,0.44,0.63}{{#1}}}
    \newcommand{\SpecialStringTok}[1]{\textcolor[rgb]{0.73,0.40,0.53}{{#1}}}
    \newcommand{\ImportTok}[1]{{#1}}
    \newcommand{\DocumentationTok}[1]{\textcolor[rgb]{0.73,0.13,0.13}{\textit{{#1}}}}
    \newcommand{\AnnotationTok}[1]{\textcolor[rgb]{0.38,0.63,0.69}{\textbf{\textit{{#1}}}}}
    \newcommand{\CommentVarTok}[1]{\textcolor[rgb]{0.38,0.63,0.69}{\textbf{\textit{{#1}}}}}
    \newcommand{\VariableTok}[1]{\textcolor[rgb]{0.10,0.09,0.49}{{#1}}}
    \newcommand{\ControlFlowTok}[1]{\textcolor[rgb]{0.00,0.44,0.13}{\textbf{{#1}}}}
    \newcommand{\OperatorTok}[1]{\textcolor[rgb]{0.40,0.40,0.40}{{#1}}}
    \newcommand{\BuiltInTok}[1]{{#1}}
    \newcommand{\ExtensionTok}[1]{{#1}}
    \newcommand{\PreprocessorTok}[1]{\textcolor[rgb]{0.74,0.48,0.00}{{#1}}}
    \newcommand{\AttributeTok}[1]{\textcolor[rgb]{0.49,0.56,0.16}{{#1}}}
    \newcommand{\InformationTok}[1]{\textcolor[rgb]{0.38,0.63,0.69}{\textbf{\textit{{#1}}}}}
    \newcommand{\WarningTok}[1]{\textcolor[rgb]{0.38,0.63,0.69}{\textbf{\textit{{#1}}}}}
    
    
    % Define a nice break command that doesn't care if a line doesn't already
    % exist.
    \def\br{\hspace*{\fill} \\* }
    % Math Jax compatability definitions
    \def\gt{>}
    \def\lt{<}
    % Document parameters
    \title{Vectors}
    
    
    

    % Pygments definitions
    
\makeatletter
\def\PY@reset{\let\PY@it=\relax \let\PY@bf=\relax%
    \let\PY@ul=\relax \let\PY@tc=\relax%
    \let\PY@bc=\relax \let\PY@ff=\relax}
\def\PY@tok#1{\csname PY@tok@#1\endcsname}
\def\PY@toks#1+{\ifx\relax#1\empty\else%
    \PY@tok{#1}\expandafter\PY@toks\fi}
\def\PY@do#1{\PY@bc{\PY@tc{\PY@ul{%
    \PY@it{\PY@bf{\PY@ff{#1}}}}}}}
\def\PY#1#2{\PY@reset\PY@toks#1+\relax+\PY@do{#2}}

\expandafter\def\csname PY@tok@ss\endcsname{\def\PY@tc##1{\textcolor[rgb]{0.10,0.09,0.49}{##1}}}
\expandafter\def\csname PY@tok@nd\endcsname{\def\PY@tc##1{\textcolor[rgb]{0.67,0.13,1.00}{##1}}}
\expandafter\def\csname PY@tok@gs\endcsname{\let\PY@bf=\textbf}
\expandafter\def\csname PY@tok@gh\endcsname{\let\PY@bf=\textbf\def\PY@tc##1{\textcolor[rgb]{0.00,0.00,0.50}{##1}}}
\expandafter\def\csname PY@tok@sr\endcsname{\def\PY@tc##1{\textcolor[rgb]{0.73,0.40,0.53}{##1}}}
\expandafter\def\csname PY@tok@sa\endcsname{\def\PY@tc##1{\textcolor[rgb]{0.73,0.13,0.13}{##1}}}
\expandafter\def\csname PY@tok@kt\endcsname{\def\PY@tc##1{\textcolor[rgb]{0.69,0.00,0.25}{##1}}}
\expandafter\def\csname PY@tok@kn\endcsname{\let\PY@bf=\textbf\def\PY@tc##1{\textcolor[rgb]{0.00,0.50,0.00}{##1}}}
\expandafter\def\csname PY@tok@cpf\endcsname{\let\PY@it=\textit\def\PY@tc##1{\textcolor[rgb]{0.25,0.50,0.50}{##1}}}
\expandafter\def\csname PY@tok@cm\endcsname{\let\PY@it=\textit\def\PY@tc##1{\textcolor[rgb]{0.25,0.50,0.50}{##1}}}
\expandafter\def\csname PY@tok@cp\endcsname{\def\PY@tc##1{\textcolor[rgb]{0.74,0.48,0.00}{##1}}}
\expandafter\def\csname PY@tok@kp\endcsname{\def\PY@tc##1{\textcolor[rgb]{0.00,0.50,0.00}{##1}}}
\expandafter\def\csname PY@tok@nf\endcsname{\def\PY@tc##1{\textcolor[rgb]{0.00,0.00,1.00}{##1}}}
\expandafter\def\csname PY@tok@vm\endcsname{\def\PY@tc##1{\textcolor[rgb]{0.10,0.09,0.49}{##1}}}
\expandafter\def\csname PY@tok@kr\endcsname{\let\PY@bf=\textbf\def\PY@tc##1{\textcolor[rgb]{0.00,0.50,0.00}{##1}}}
\expandafter\def\csname PY@tok@vc\endcsname{\def\PY@tc##1{\textcolor[rgb]{0.10,0.09,0.49}{##1}}}
\expandafter\def\csname PY@tok@mf\endcsname{\def\PY@tc##1{\textcolor[rgb]{0.40,0.40,0.40}{##1}}}
\expandafter\def\csname PY@tok@gu\endcsname{\let\PY@bf=\textbf\def\PY@tc##1{\textcolor[rgb]{0.50,0.00,0.50}{##1}}}
\expandafter\def\csname PY@tok@gp\endcsname{\let\PY@bf=\textbf\def\PY@tc##1{\textcolor[rgb]{0.00,0.00,0.50}{##1}}}
\expandafter\def\csname PY@tok@s\endcsname{\def\PY@tc##1{\textcolor[rgb]{0.73,0.13,0.13}{##1}}}
\expandafter\def\csname PY@tok@cs\endcsname{\let\PY@it=\textit\def\PY@tc##1{\textcolor[rgb]{0.25,0.50,0.50}{##1}}}
\expandafter\def\csname PY@tok@k\endcsname{\let\PY@bf=\textbf\def\PY@tc##1{\textcolor[rgb]{0.00,0.50,0.00}{##1}}}
\expandafter\def\csname PY@tok@mi\endcsname{\def\PY@tc##1{\textcolor[rgb]{0.40,0.40,0.40}{##1}}}
\expandafter\def\csname PY@tok@fm\endcsname{\def\PY@tc##1{\textcolor[rgb]{0.00,0.00,1.00}{##1}}}
\expandafter\def\csname PY@tok@mb\endcsname{\def\PY@tc##1{\textcolor[rgb]{0.40,0.40,0.40}{##1}}}
\expandafter\def\csname PY@tok@gd\endcsname{\def\PY@tc##1{\textcolor[rgb]{0.63,0.00,0.00}{##1}}}
\expandafter\def\csname PY@tok@gi\endcsname{\def\PY@tc##1{\textcolor[rgb]{0.00,0.63,0.00}{##1}}}
\expandafter\def\csname PY@tok@vg\endcsname{\def\PY@tc##1{\textcolor[rgb]{0.10,0.09,0.49}{##1}}}
\expandafter\def\csname PY@tok@nl\endcsname{\def\PY@tc##1{\textcolor[rgb]{0.63,0.63,0.00}{##1}}}
\expandafter\def\csname PY@tok@ne\endcsname{\let\PY@bf=\textbf\def\PY@tc##1{\textcolor[rgb]{0.82,0.25,0.23}{##1}}}
\expandafter\def\csname PY@tok@sb\endcsname{\def\PY@tc##1{\textcolor[rgb]{0.73,0.13,0.13}{##1}}}
\expandafter\def\csname PY@tok@kc\endcsname{\let\PY@bf=\textbf\def\PY@tc##1{\textcolor[rgb]{0.00,0.50,0.00}{##1}}}
\expandafter\def\csname PY@tok@s2\endcsname{\def\PY@tc##1{\textcolor[rgb]{0.73,0.13,0.13}{##1}}}
\expandafter\def\csname PY@tok@na\endcsname{\def\PY@tc##1{\textcolor[rgb]{0.49,0.56,0.16}{##1}}}
\expandafter\def\csname PY@tok@nn\endcsname{\let\PY@bf=\textbf\def\PY@tc##1{\textcolor[rgb]{0.00,0.00,1.00}{##1}}}
\expandafter\def\csname PY@tok@c\endcsname{\let\PY@it=\textit\def\PY@tc##1{\textcolor[rgb]{0.25,0.50,0.50}{##1}}}
\expandafter\def\csname PY@tok@w\endcsname{\def\PY@tc##1{\textcolor[rgb]{0.73,0.73,0.73}{##1}}}
\expandafter\def\csname PY@tok@s1\endcsname{\def\PY@tc##1{\textcolor[rgb]{0.73,0.13,0.13}{##1}}}
\expandafter\def\csname PY@tok@dl\endcsname{\def\PY@tc##1{\textcolor[rgb]{0.73,0.13,0.13}{##1}}}
\expandafter\def\csname PY@tok@sh\endcsname{\def\PY@tc##1{\textcolor[rgb]{0.73,0.13,0.13}{##1}}}
\expandafter\def\csname PY@tok@mo\endcsname{\def\PY@tc##1{\textcolor[rgb]{0.40,0.40,0.40}{##1}}}
\expandafter\def\csname PY@tok@ow\endcsname{\let\PY@bf=\textbf\def\PY@tc##1{\textcolor[rgb]{0.67,0.13,1.00}{##1}}}
\expandafter\def\csname PY@tok@no\endcsname{\def\PY@tc##1{\textcolor[rgb]{0.53,0.00,0.00}{##1}}}
\expandafter\def\csname PY@tok@bp\endcsname{\def\PY@tc##1{\textcolor[rgb]{0.00,0.50,0.00}{##1}}}
\expandafter\def\csname PY@tok@mh\endcsname{\def\PY@tc##1{\textcolor[rgb]{0.40,0.40,0.40}{##1}}}
\expandafter\def\csname PY@tok@nv\endcsname{\def\PY@tc##1{\textcolor[rgb]{0.10,0.09,0.49}{##1}}}
\expandafter\def\csname PY@tok@go\endcsname{\def\PY@tc##1{\textcolor[rgb]{0.53,0.53,0.53}{##1}}}
\expandafter\def\csname PY@tok@ge\endcsname{\let\PY@it=\textit}
\expandafter\def\csname PY@tok@kd\endcsname{\let\PY@bf=\textbf\def\PY@tc##1{\textcolor[rgb]{0.00,0.50,0.00}{##1}}}
\expandafter\def\csname PY@tok@vi\endcsname{\def\PY@tc##1{\textcolor[rgb]{0.10,0.09,0.49}{##1}}}
\expandafter\def\csname PY@tok@err\endcsname{\def\PY@bc##1{\setlength{\fboxsep}{0pt}\fcolorbox[rgb]{1.00,0.00,0.00}{1,1,1}{\strut ##1}}}
\expandafter\def\csname PY@tok@o\endcsname{\def\PY@tc##1{\textcolor[rgb]{0.40,0.40,0.40}{##1}}}
\expandafter\def\csname PY@tok@c1\endcsname{\let\PY@it=\textit\def\PY@tc##1{\textcolor[rgb]{0.25,0.50,0.50}{##1}}}
\expandafter\def\csname PY@tok@sc\endcsname{\def\PY@tc##1{\textcolor[rgb]{0.73,0.13,0.13}{##1}}}
\expandafter\def\csname PY@tok@nt\endcsname{\let\PY@bf=\textbf\def\PY@tc##1{\textcolor[rgb]{0.00,0.50,0.00}{##1}}}
\expandafter\def\csname PY@tok@il\endcsname{\def\PY@tc##1{\textcolor[rgb]{0.40,0.40,0.40}{##1}}}
\expandafter\def\csname PY@tok@gr\endcsname{\def\PY@tc##1{\textcolor[rgb]{1.00,0.00,0.00}{##1}}}
\expandafter\def\csname PY@tok@ni\endcsname{\let\PY@bf=\textbf\def\PY@tc##1{\textcolor[rgb]{0.60,0.60,0.60}{##1}}}
\expandafter\def\csname PY@tok@ch\endcsname{\let\PY@it=\textit\def\PY@tc##1{\textcolor[rgb]{0.25,0.50,0.50}{##1}}}
\expandafter\def\csname PY@tok@nb\endcsname{\def\PY@tc##1{\textcolor[rgb]{0.00,0.50,0.00}{##1}}}
\expandafter\def\csname PY@tok@sd\endcsname{\let\PY@it=\textit\def\PY@tc##1{\textcolor[rgb]{0.73,0.13,0.13}{##1}}}
\expandafter\def\csname PY@tok@nc\endcsname{\let\PY@bf=\textbf\def\PY@tc##1{\textcolor[rgb]{0.00,0.00,1.00}{##1}}}
\expandafter\def\csname PY@tok@sx\endcsname{\def\PY@tc##1{\textcolor[rgb]{0.00,0.50,0.00}{##1}}}
\expandafter\def\csname PY@tok@si\endcsname{\let\PY@bf=\textbf\def\PY@tc##1{\textcolor[rgb]{0.73,0.40,0.53}{##1}}}
\expandafter\def\csname PY@tok@m\endcsname{\def\PY@tc##1{\textcolor[rgb]{0.40,0.40,0.40}{##1}}}
\expandafter\def\csname PY@tok@se\endcsname{\let\PY@bf=\textbf\def\PY@tc##1{\textcolor[rgb]{0.73,0.40,0.13}{##1}}}
\expandafter\def\csname PY@tok@gt\endcsname{\def\PY@tc##1{\textcolor[rgb]{0.00,0.27,0.87}{##1}}}

\def\PYZbs{\char`\\}
\def\PYZus{\char`\_}
\def\PYZob{\char`\{}
\def\PYZcb{\char`\}}
\def\PYZca{\char`\^}
\def\PYZam{\char`\&}
\def\PYZlt{\char`\<}
\def\PYZgt{\char`\>}
\def\PYZsh{\char`\#}
\def\PYZpc{\char`\%}
\def\PYZdl{\char`\$}
\def\PYZhy{\char`\-}
\def\PYZsq{\char`\'}
\def\PYZdq{\char`\"}
\def\PYZti{\char`\~}
% for compatibility with earlier versions
\def\PYZat{@}
\def\PYZlb{[}
\def\PYZrb{]}
\makeatother


    % Exact colors from NB
    \definecolor{incolor}{rgb}{0.0, 0.0, 0.5}
    \definecolor{outcolor}{rgb}{0.545, 0.0, 0.0}



    
    % Prevent overflowing lines due to hard-to-break entities
    \sloppy 
    % Setup hyperref package
    \hypersetup{
      breaklinks=true,  % so long urls are correctly broken across lines
      colorlinks=true,
      urlcolor=urlcolor,
      linkcolor=linkcolor,
      citecolor=citecolor,
      }
    % Slightly bigger margins than the latex defaults
    
    \geometry{verbose,tmargin=1in,bmargin=1in,lmargin=1in,rmargin=1in}
    
    

    \begin{document}
    
    
    \maketitle
    
    

    
    \begin{Verbatim}[commandchars=\\\{\}]
{\color{incolor}In [{\color{incolor}15}]:} \PY{k+kn}{from} \PY{n+nn}{IPython}\PY{n+nn}{.}\PY{n+nn}{display} \PY{k}{import} \PY{n}{HTML}
         
         \PY{n}{HTML}\PY{p}{(}\PY{l+s+s1}{\PYZsq{}\PYZsq{}\PYZsq{}}\PY{l+s+s1}{\PYZlt{}script\PYZgt{}}
         \PY{l+s+s1}{code\PYZus{}show=true; }
         \PY{l+s+s1}{function code\PYZus{}toggle() }\PY{l+s+s1}{\PYZob{}}
         \PY{l+s+s1}{ if (code\PYZus{}show)}\PY{l+s+s1}{\PYZob{}}
         \PY{l+s+s1}{ \PYZdl{}(}\PY{l+s+s1}{\PYZsq{}}\PY{l+s+s1}{div.input:contains(}\PY{l+s+s1}{\PYZdq{}}\PY{l+s+si}{\PYZpc{}\PYZpc{}}\PY{l+s+s1}{latex}\PY{l+s+s1}{\PYZdq{}}\PY{l+s+s1}{)}\PY{l+s+s1}{\PYZsq{}}\PY{l+s+s1}{).hide();}
         \PY{l+s+s1}{ //alert(}\PY{l+s+s1}{\PYZdq{}}\PY{l+s+s1}{Dela!}\PY{l+s+s1}{\PYZdq{}}\PY{l+s+s1}{);}
         \PY{l+s+s1}{ \PYZcb{} else }\PY{l+s+s1}{\PYZob{}}
         \PY{l+s+s1}{ \PYZdl{}(}\PY{l+s+s1}{\PYZsq{}}\PY{l+s+s1}{div.input:contains(}\PY{l+s+s1}{\PYZdq{}}\PY{l+s+si}{\PYZpc{}\PYZpc{}}\PY{l+s+s1}{latex}\PY{l+s+s1}{\PYZdq{}}\PY{l+s+s1}{)}\PY{l+s+s1}{\PYZsq{}}\PY{l+s+s1}{).show();}
         \PY{l+s+s1}{ \PYZcb{}}
         \PY{l+s+s1}{ code\PYZus{}show = !code\PYZus{}show}
         \PY{l+s+s1}{\PYZcb{} }
         \PY{l+s+s1}{\PYZdl{}( document ).ready(code\PYZus{}toggle);}
         \PY{l+s+s1}{\PYZlt{}/script\PYZgt{}}
         \PY{l+s+s1}{\PYZlt{}form action=}\PY{l+s+s1}{\PYZdq{}}\PY{l+s+s1}{javascript:code\PYZus{}toggle()}\PY{l+s+s1}{\PYZdq{}}\PY{l+s+s1}{\PYZgt{}\PYZlt{}input type=}\PY{l+s+s1}{\PYZdq{}}\PY{l+s+s1}{submit}\PY{l+s+s1}{\PYZdq{}}\PY{l+s+s1}{ value=}\PY{l+s+s1}{\PYZdq{}}\PY{l+s+s1}{Toggle on/off the LaTeX code.}\PY{l+s+s1}{\PYZdq{}}\PY{l+s+s1}{\PYZgt{}\PYZlt{}/form\PYZgt{}}\PY{l+s+s1}{\PYZsq{}\PYZsq{}\PYZsq{}}\PY{p}{)}
\end{Verbatim}


\begin{Verbatim}[commandchars=\\\{\}]
{\color{outcolor}Out[{\color{outcolor}15}]:} <IPython.core.display.HTML object>
\end{Verbatim}
            
    \hypertarget{vectors}{%
\section{Vectors}\label{vectors}}

    \begin{Verbatim}[commandchars=\\\{\}]
{\color{incolor}In [{\color{incolor}9}]:} \PY{c}{\PYZpc{}\PYZpc{}latex}
        \PY{l+s}{\PYZdl{}}\PY{n+nv}{\PYZbs{}mathbf}\PY{n+nb}{\PYZob{}}\PY{n+nb}{Definition:}\PY{n+nb}{\PYZcb{}}\PY{l+s}{\PYZdl{}} A vector is the ordered \PY{l+s}{\PYZdl{}}\PY{n+nb}{n}\PY{l+s}{\PYZdl{}}\PYZhy{}number of numbers (list of numbers) that are usually written as a column:
        \PY{l+s+sb}{\PYZdl{}\PYZdl{}}\PY{l+s}{\PYZdl{}\PYZdl{}}
        
        \PY{l+s+sb}{\PYZdl{}\PYZdl{}}
        \PY{n+nb}{x }\PY{o}{=}\PY{n+nb}{ }\PY{n+nv}{\PYZbs{}begin}\PY{n+nb}{\PYZob{}}\PY{n+nb}{bmatrix}\PY{n+nb}{\PYZcb{}}\PY{n+nb}{x}\PY{n+nb}{\PYZus{}}\PY{l+m}{1}\PY{n+nb}{ }\PY{n+nv}{\PYZbs{}\PYZbs{}}\PY{n+nb}{ x}\PY{n+nb}{\PYZus{}}\PY{l+m}{2}\PY{n+nb}{ }\PY{n+nv}{\PYZbs{}\PYZbs{}}\PY{n+nb}{ ... }\PY{n+nv}{\PYZbs{}\PYZbs{}}\PY{n+nb}{ x}\PY{n+nb}{\PYZus{}}\PY{n+nb}{n}\PY{n+nv}{\PYZbs{}end}\PY{n+nb}{\PYZob{}}\PY{n+nb}{bmatrix}\PY{n+nb}{\PYZcb{}}
        \PY{l+s}{\PYZdl{}\PYZdl{}}
        
        \PY{l+s+sb}{\PYZdl{}\PYZdl{}}\PY{l+s}{\PYZdl{}\PYZdl{}}
        Numbers \PY{l+s}{\PYZdl{}}\PY{n+nb}{x}\PY{n+nb}{\PYZus{}}\PY{l+m}{1}\PY{n+nb}{, ... , x}\PY{n+nb}{\PYZus{}}\PY{n+nb}{n}\PY{l+s}{\PYZdl{}} are coordinates or components of vector \PY{l+s}{\PYZdl{}}\PY{n+nb}{x}\PY{l+s}{\PYZdl{}}. 
        \PY{l+s+sb}{\PYZdl{}\PYZdl{}}\PY{l+s}{\PYZdl{}\PYZdl{}}
        Components of a vector are usually real or complex numbers. 
        In the first case there are vectors elements of the \PY{l+s}{\PYZdl{}}\PY{n+nv}{\PYZbs{}mathbb}\PY{n+nb}{\PYZob{}}\PY{n+nb}{R}\PY{n+nb}{\PYZcb{}}\PY{n+nb}{\PYZca{}}\PY{n+nb}{n}\PY{l+s}{\PYZdl{}}, in the second \PY{l+s}{\PYZdl{}}\PY{n+nv}{\PYZbs{}mathbb}\PY{n+nb}{\PYZob{}}\PY{n+nb}{C}\PY{n+nb}{\PYZcb{}}\PY{n+nb}{\PYZca{}}\PY{n+nb}{n}\PY{l+s}{\PYZdl{}}.
\end{Verbatim}


    $\mathbf{Definition:}$ A vector is the ordered $n$-number of numbers (list of numbers) that are usually written as a column:
$$$$

$$
x = \begin{bmatrix}x_1 \\ x_2 \\ ... \\ x_n\end{bmatrix}
$$

$$$$
Numbers $x_1, ... , x_n$ are coordinates or components of vector $x$. 
$$$$
Components of a vector are usually real or complex numbers. 
In the first case there are vectors elements of the $\mathbb{R}^n$, in the second $\mathbb{C}^n$.

    
    \begin{Verbatim}[commandchars=\\\{\}]
{\color{incolor}In [{\color{incolor}49}]:} \PY{c}{\PYZpc{}\PYZpc{}latex}
         We can also represent vectors as directed lines in the \PY{l+s}{\PYZdl{}}\PY{n+nb}{n}\PY{l+s}{\PYZdl{}}\PYZhy{}dimensional
         space.
         \PY{l+s+sb}{\PYZdl{}\PYZdl{}}\PY{l+s}{\PYZdl{}\PYZdl{}}
         The two pointed lines are the same when they have the same direction and the same
         length, and their starting points may be different. Thus, vectors from the set
         \PY{l+s}{\PYZdl{}}\PY{n+nv}{\PYZbs{}mathbb}\PY{n+nb}{\PYZob{}}\PY{n+nb}{R}\PY{n+nb}{\PYZcb{}}\PY{n+nb}{\PYZca{}}\PY{l+m}{2}\PY{l+s}{\PYZdl{}} is plotted in the coordinate plane so that the directed line runs from the coordinate plane
         starting point from point A, the coordinates of which are the components of the vector.
\end{Verbatim}


    We can also represent vectors as directed lines in the $n$-dimensional
space.
$$$$
The two pointed lines are the same when they have the same direction and the same
length, and their starting points may be different. Thus, vectors from the set
$\mathbb{R}^2$ is plotted in the coordinate plane so that the directed line runs from the coordinate plane
starting point from point A, the coordinates of which are the components of the vector.

    
    \hypertarget{operations-with-vectors}{%
\subsection{Operations with vectors}\label{operations-with-vectors}}

\hypertarget{product-of-scalar-and-vector}{%
\subsubsection{Product of scalar and
vector}\label{product-of-scalar-and-vector}}

    \begin{Verbatim}[commandchars=\\\{\}]
{\color{incolor}In [{\color{incolor}121}]:} \PY{c}{\PYZpc{}\PYZpc{}latex}
          \PY{l+s}{\PYZdl{}}\PY{n+nv}{\PYZbs{}mathbf}\PY{n+nb}{\PYZob{}}\PY{n+nb}{Definition:}\PY{n+nb}{\PYZcb{}}\PY{l+s}{\PYZdl{}} The product of vector \PY{l+s}{\PYZdl{}}\PY{n+nb}{x}\PY{l+s}{\PYZdl{}} with scalar \PY{l+s}{\PYZdl{}}\PY{n+nv}{\PYZbs{}alpha}\PY{l+s}{\PYZdl{}} is a vector
          \PY{l+s+sb}{\PYZdl{}\PYZdl{}}\PY{l+s}{\PYZdl{}\PYZdl{}}
          \PY{l+s+sb}{\PYZdl{}\PYZdl{}}
          \PY{n+nv}{\PYZbs{}alpha}\PY{n+nb}{ x }\PY{o}{=}\PY{n+nb}{ }\PY{n+nv}{\PYZbs{}alpha}\PY{n+nb}{ }\PY{n+nv}{\PYZbs{}begin}\PY{n+nb}{\PYZob{}}\PY{n+nb}{bmatrix}\PY{n+nb}{\PYZcb{}}\PY{n+nb}{x}\PY{n+nb}{\PYZus{}}\PY{l+m}{1}\PY{n+nb}{ }\PY{n+nv}{\PYZbs{}\PYZbs{}}\PY{n+nb}{ x}\PY{n+nb}{\PYZus{}}\PY{l+m}{2}\PY{n+nb}{ }\PY{n+nv}{\PYZbs{}\PYZbs{}}\PY{n+nb}{ ... }\PY{n+nv}{\PYZbs{}\PYZbs{}}\PY{n+nb}{ x}\PY{n+nb}{\PYZus{}}\PY{n+nb}{n}\PY{n+nv}{\PYZbs{}end}\PY{n+nb}{\PYZob{}}\PY{n+nb}{bmatrix}\PY{n+nb}{\PYZcb{}}\PY{n+nb}{ }\PY{o}{=}\PY{n+nb}{ }
          \PY{n+nb}{            }\PY{n+nv}{\PYZbs{}begin}\PY{n+nb}{\PYZob{}}\PY{n+nb}{bmatrix}\PY{n+nb}{\PYZcb{}}\PY{n+nb}{ }\PY{n+nv}{\PYZbs{}alpha}\PY{n+nb}{ x}\PY{n+nb}{\PYZus{}}\PY{l+m}{1}\PY{n+nb}{ }\PY{n+nv}{\PYZbs{}\PYZbs{}}\PY{n+nb}{ }\PY{n+nv}{\PYZbs{}alpha}\PY{n+nb}{ x}\PY{n+nb}{\PYZus{}}\PY{l+m}{2}\PY{n+nb}{ }\PY{n+nv}{\PYZbs{}\PYZbs{}}\PY{n+nb}{ ... }\PY{n+nv}{\PYZbs{}\PYZbs{}}\PY{n+nb}{ }\PY{n+nv}{\PYZbs{}alpha}\PY{n+nb}{ x}\PY{n+nb}{\PYZus{}}\PY{n+nb}{n}\PY{n+nv}{\PYZbs{}end}\PY{n+nb}{\PYZob{}}\PY{n+nb}{bmatrix}\PY{n+nb}{\PYZcb{}}
          \PY{l+s}{\PYZdl{}\PYZdl{}}
          
          \PY{l+s+sb}{\PYZdl{}\PYZdl{}}\PY{l+s}{\PYZdl{}\PYZdl{}}
          
          Vectors \PY{l+s}{\PYZdl{}}\PY{n+nb}{a}\PY{l+s}{\PYZdl{}} and \PY{l+s}{\PYZdl{}}\PY{n+nb}{b}\PY{l+s}{\PYZdl{}} for which there is such a scalar \PY{l+s}{\PYZdl{}}\PY{n+nv}{\PYZbs{}alpha}\PY{l+s}{\PYZdl{}} that \PY{l+s}{\PYZdl{}}\PY{n+nb}{a }\PY{o}{=}\PY{n+nb}{ }\PY{n+nv}{\PYZbs{}alpha}\PY{n+nb}{ b}\PY{l+s}{\PYZdl{}} or \PY{l+s}{\PYZdl{}}\PY{n+nb}{b }\PY{o}{=}\PY{n+nb}{ }\PY{n+nv}{\PYZbs{}alpha}\PY{n+nb}{ a}\PY{l+s}{\PYZdl{}}, are
          called \PY{l+s}{\PYZdl{}}\PY{n+nv}{\PYZbs{}mathbf}\PY{n+nb}{\PYZob{}}\PY{n+nb}{collinear\PYZti{}vectors}\PY{n+nb}{\PYZcb{}}\PY{l+s}{\PYZdl{}}. Collinear vectors (as the name suggests) lie on parallel lines.
\end{Verbatim}


    $\mathbf{Definition:}$ The product of vector $x$ with scalar $\alpha$ is a vector
$$$$
$$
\alpha x = \alpha \begin{bmatrix}x_1 \\ x_2 \\ ... \\ x_n\end{bmatrix} = 
            \begin{bmatrix} \alpha x_1 \\ \alpha x_2 \\ ... \\ \alpha x_n\end{bmatrix}
$$

$$$$

Vectors $a$ and $b$ for which there is such a scalar $\alpha$ that $a = \alpha b$ or $b = \alpha a$, are
called $\mathbf{collinear~vectors}$. Collinear vectors (as the name suggests) lie on parallel lines.

    
    \hypertarget{sum-of-vectors}{%
\subsubsection{Sum of vectors}\label{sum-of-vectors}}

    \begin{Verbatim}[commandchars=\\\{\}]
{\color{incolor}In [{\color{incolor}132}]:} \PY{c}{\PYZpc{}\PYZpc{}latex}
          \PY{l+s}{\PYZdl{}}\PY{n+nv}{\PYZbs{}mathbf}\PY{n+nb}{\PYZob{}}\PY{n+nb}{Definition:}\PY{n+nb}{\PYZcb{}}\PY{l+s}{\PYZdl{}} The sum of the vectors \PY{l+s}{\PYZdl{}}\PY{n+nb}{x}\PY{l+s}{\PYZdl{}} and \PY{l+s}{\PYZdl{}}\PY{n+nb}{y}\PY{l+s}{\PYZdl{}} is a vector
          \PY{l+s+sb}{\PYZdl{}\PYZdl{}}\PY{l+s}{\PYZdl{}\PYZdl{}}
          \PY{l+s+sb}{\PYZdl{}\PYZdl{}}
          \PY{n+nb}{ x }\PY{o}{+}\PY{n+nb}{ y }\PY{o}{=}\PY{n+nb}{ }\PY{n+nv}{\PYZbs{}begin}\PY{n+nb}{\PYZob{}}\PY{n+nb}{bmatrix}\PY{n+nb}{\PYZcb{}}\PY{n+nb}{x}\PY{n+nb}{\PYZus{}}\PY{l+m}{1}\PY{n+nb}{ }\PY{n+nv}{\PYZbs{}\PYZbs{}}\PY{n+nb}{ x}\PY{n+nb}{\PYZus{}}\PY{l+m}{2}\PY{n+nb}{ }\PY{n+nv}{\PYZbs{}\PYZbs{}}\PY{n+nb}{ ... }\PY{n+nv}{\PYZbs{}\PYZbs{}}\PY{n+nb}{ x}\PY{n+nb}{\PYZus{}}\PY{n+nb}{n}\PY{n+nv}{\PYZbs{}end}\PY{n+nb}{\PYZob{}}\PY{n+nb}{bmatrix}\PY{n+nb}{\PYZcb{}}\PY{n+nb}{ }\PY{o}{+}\PY{n+nb}{ }
          \PY{n+nb}{         }\PY{n+nv}{\PYZbs{}begin}\PY{n+nb}{\PYZob{}}\PY{n+nb}{bmatrix}\PY{n+nb}{\PYZcb{}}\PY{n+nb}{y}\PY{n+nb}{\PYZus{}}\PY{l+m}{1}\PY{n+nb}{ }\PY{n+nv}{\PYZbs{}\PYZbs{}}\PY{n+nb}{ y}\PY{n+nb}{\PYZus{}}\PY{l+m}{2}\PY{n+nb}{ }\PY{n+nv}{\PYZbs{}\PYZbs{}}\PY{n+nb}{ ... }\PY{n+nv}{\PYZbs{}\PYZbs{}}\PY{n+nb}{ y}\PY{n+nb}{\PYZus{}}\PY{n+nb}{n}\PY{n+nv}{\PYZbs{}end}\PY{n+nb}{\PYZob{}}\PY{n+nb}{bmatrix}\PY{n+nb}{\PYZcb{}}\PY{n+nb}{ }\PY{o}{=}\PY{n+nb}{ }
          \PY{n+nb}{         }\PY{n+nv}{\PYZbs{}begin}\PY{n+nb}{\PYZob{}}\PY{n+nb}{bmatrix}\PY{n+nb}{\PYZcb{}}\PY{n+nb}{ x}\PY{n+nb}{\PYZus{}}\PY{l+m}{1}\PY{n+nb}{ }\PY{o}{+}\PY{n+nb}{ y}\PY{n+nb}{\PYZus{}}\PY{l+m}{1}\PY{n+nb}{ }\PY{n+nv}{\PYZbs{}\PYZbs{}}\PY{n+nb}{ x}\PY{n+nb}{\PYZus{}}\PY{l+m}{2}\PY{n+nb}{ }\PY{o}{+}\PY{n+nb}{ y}\PY{n+nb}{\PYZus{}}\PY{l+m}{2}\PY{n+nb}{ }\PY{n+nv}{\PYZbs{}\PYZbs{}}\PY{n+nb}{ ... }\PY{n+nv}{\PYZbs{}\PYZbs{}}\PY{n+nb}{ x}\PY{n+nb}{\PYZus{}}\PY{n+nb}{n }\PY{o}{+}\PY{n+nb}{ y}\PY{n+nb}{\PYZus{}}\PY{n+nb}{n }\PY{n+nv}{\PYZbs{}end}\PY{n+nb}{\PYZob{}}\PY{n+nb}{bmatrix}\PY{n+nb}{\PYZcb{}}
          \PY{l+s}{\PYZdl{}\PYZdl{}}
          \PY{l+s+sb}{\PYZdl{}\PYZdl{}}\PY{l+s}{\PYZdl{}\PYZdl{}}
          \PY{l+s}{\PYZdl{}}\PY{n+nv}{\PYZbs{}mathbf}\PY{n+nb}{\PYZob{}}\PY{n+nb}{Attention}\PY{o}{!}\PY{n+nb}{\PYZcb{}}\PY{l+s}{\PYZdl{}} The vectors that we add together must have the same number of components!
\end{Verbatim}


    $\mathbf{Definition:}$ The sum of the vectors $x$ and $y$ is a vector
$$$$
$$
 x + y = \begin{bmatrix}x_1 \\ x_2 \\ ... \\ x_n\end{bmatrix} + 
         \begin{bmatrix}y_1 \\ y_2 \\ ... \\ y_n\end{bmatrix} = 
         \begin{bmatrix} x_1 + y_1 \\ x_2 + y_2 \\ ... \\ x_n + y_n \end{bmatrix}
$$
$$$$
$\mathbf{Attention!}$ The vectors that we add together must have the same number of components!

    
    \hypertarget{null-vector}{%
\subsubsection{Null vector}\label{null-vector}}

    \begin{Verbatim}[commandchars=\\\{\}]
{\color{incolor}In [{\color{incolor}85}]:} \PY{c}{\PYZpc{}\PYZpc{}latex}
         \PY{l+s}{\PYZdl{}}\PY{n+nv}{\PYZbs{}mathbf}\PY{n+nb}{\PYZob{}}\PY{n+nb}{Definition:}\PY{n+nb}{\PYZcb{}}\PY{l+s}{\PYZdl{}} The zero vector \PY{l+s}{\PYZdl{}}\PY{l+m}{0}\PY{l+s}{\PYZdl{}} is the vector for which \PY{l+s}{\PYZdl{}}\PY{n+nb}{a }\PY{o}{+}\PY{n+nb}{ }\PY{l+m}{0}\PY{n+nb}{ }\PY{o}{=}\PY{n+nb}{ }\PY{l+m}{0}\PY{n+nb}{ }\PY{o}{+}\PY{n+nb}{ a }\PY{o}{=}\PY{n+nb}{ a}\PY{l+s}{\PYZdl{}}
         for each vector \PY{l+s}{\PYZdl{}}\PY{n+nb}{a}\PY{l+s}{\PYZdl{}}. All components of the zero vector are equal to \PY{l+s}{\PYZdl{}}\PY{l+m}{0}\PY{l+s}{\PYZdl{}}. Each
         the vector a belongs to the opposite vector \PY{l+s}{\PYZdl{}}\PY{o}{\PYZhy{}}\PY{n+nb}{a}\PY{l+s}{\PYZdl{}}, so \PY{l+s}{\PYZdl{}}\PY{n+nb}{a }\PY{o}{+}\PY{n+nb}{ }\PY{o}{(}\PY{o}{\PYZhy{}}\PY{n+nb}{a}\PY{o}{)}\PY{n+nb}{ }\PY{o}{=}\PY{n+nb}{ }\PY{l+m}{0}\PY{l+s}{\PYZdl{}}. The difference
         the vectors \PY{l+s}{\PYZdl{}}\PY{n+nb}{a}\PY{l+s}{\PYZdl{}} and \PY{l+s}{\PYZdl{}}\PY{n+nb}{b}\PY{l+s}{\PYZdl{}} are the sum of \PY{l+s}{\PYZdl{}}\PY{n+nb}{a }\PY{o}{+}\PY{n+nb}{ }\PY{o}{(}\PY{o}{\PYZhy{}}\PY{n+nb}{b}\PY{o}{)}\PY{l+s}{\PYZdl{}} and are usually written as \PY{l+s}{\PYZdl{}}\PY{n+nb}{a }\PY{o}{\PYZhy{}}\PY{n+nb}{ b}\PY{l+s}{\PYZdl{}}.
         \PY{l+s+sb}{\PYZdl{}\PYZdl{}}\PY{l+s}{\PYZdl{}\PYZdl{}}
         \PY{l+s+sb}{\PYZdl{}\PYZdl{}}
         \PY{n+nb}{ a }\PY{o}{+}\PY{n+nb}{ }\PY{l+m}{0}\PY{n+nb}{ }\PY{o}{=}\PY{n+nb}{ }\PY{n+nv}{\PYZbs{}begin}\PY{n+nb}{\PYZob{}}\PY{n+nb}{bmatrix}\PY{n+nb}{\PYZcb{}}\PY{n+nb}{a}\PY{n+nb}{\PYZus{}}\PY{l+m}{1}\PY{n+nb}{ }\PY{n+nv}{\PYZbs{}\PYZbs{}}\PY{n+nb}{ a}\PY{n+nb}{\PYZus{}}\PY{l+m}{2}\PY{n+nb}{ }\PY{n+nv}{\PYZbs{}\PYZbs{}}\PY{n+nb}{ ... }\PY{n+nv}{\PYZbs{}\PYZbs{}}\PY{n+nb}{ a}\PY{n+nb}{\PYZus{}}\PY{n+nb}{n}\PY{n+nv}{\PYZbs{}end}\PY{n+nb}{\PYZob{}}\PY{n+nb}{bmatrix}\PY{n+nb}{\PYZcb{}}\PY{n+nb}{ }\PY{o}{+}\PY{n+nb}{ }
         \PY{n+nb}{         }\PY{n+nv}{\PYZbs{}begin}\PY{n+nb}{\PYZob{}}\PY{n+nb}{bmatrix}\PY{n+nb}{\PYZcb{}}\PY{n+nb}{ }\PY{l+m}{0}\PY{n+nb}{ }\PY{n+nv}{\PYZbs{}\PYZbs{}}\PY{n+nb}{ }\PY{l+m}{0}\PY{n+nb}{ }\PY{n+nv}{\PYZbs{}\PYZbs{}}\PY{n+nb}{ ... }\PY{n+nv}{\PYZbs{}\PYZbs{}}\PY{n+nb}{ }\PY{l+m}{0}\PY{n+nv}{\PYZbs{}end}\PY{n+nb}{\PYZob{}}\PY{n+nb}{bmatrix}\PY{n+nb}{\PYZcb{}}\PY{n+nb}{ }\PY{o}{=}\PY{n+nb}{ }
         \PY{n+nb}{         }\PY{n+nv}{\PYZbs{}begin}\PY{n+nb}{\PYZob{}}\PY{n+nb}{bmatrix}\PY{n+nb}{\PYZcb{}}\PY{n+nb}{ a}\PY{n+nb}{\PYZus{}}\PY{l+m}{1}\PY{n+nb}{ }\PY{n+nv}{\PYZbs{}\PYZbs{}}\PY{n+nb}{ a}\PY{n+nb}{\PYZus{}}\PY{l+m}{2}\PY{n+nb}{ }\PY{n+nv}{\PYZbs{}\PYZbs{}}\PY{n+nb}{ ... }\PY{n+nv}{\PYZbs{}\PYZbs{}}\PY{n+nb}{ a}\PY{n+nb}{\PYZus{}}\PY{n+nb}{n}\PY{n+nv}{\PYZbs{}end}\PY{n+nb}{\PYZob{}}\PY{n+nb}{bmatrix}\PY{n+nb}{\PYZcb{}}\PY{n+nb}{ }\PY{o}{=}\PY{n+nb}{ }\PY{l+m}{0}\PY{n+nb}{ }\PY{o}{+}\PY{n+nb}{ a }\PY{o}{=}\PY{n+nb}{ a}
         \PY{l+s}{\PYZdl{}\PYZdl{}}
\end{Verbatim}


    $\mathbf{Definition:}$ The zero vector $0$ is the vector for which $a + 0 = 0 + a = a$
for each vector $a$. All components of the zero vector are equal to $0$. Each
the vector a belongs to the opposite vector $-a$, so $a + (-a) = 0$. The difference
the vectors $a$ and $b$ are the sum of $a + (-b)$ and are usually written as $a - b$.
$$$$
$$
 a + 0 = \begin{bmatrix}a_1 \\ a_2 \\ ... \\ a_n\end{bmatrix} + 
         \begin{bmatrix} 0 \\ 0 \\ ... \\ 0\end{bmatrix} = 
         \begin{bmatrix} a_1 \\ a_2 \\ ... \\ a_n\end{bmatrix} = 0 + a = a
$$

    
    \hypertarget{properties-of-vector-sum-and-product-with-scalar}{%
\subsubsection{Properties of vector sum and product with
scalar}\label{properties-of-vector-sum-and-product-with-scalar}}

    \begin{Verbatim}[commandchars=\\\{\}]
{\color{incolor}In [{\color{incolor}103}]:} \PY{c}{\PYZpc{}\PYZpc{}latex}
          The vector sum and product of a vector with a scalar have some similar properties
          as the sum and product of the numbers.
          \PY{l+s+sb}{\PYZdl{}\PYZdl{}}\PY{l+s}{\PYZdl{}\PYZdl{}}
          \PY{l+s}{\PYZdl{}}\PY{n+nv}{\PYZbs{}mathbf}\PY{n+nb}{\PYZob{}}\PY{n+nb}{Property\PYZti{}}\PY{l+m}{1}\PY{n+nb}{:}\PY{n+nb}{\PYZcb{}}\PY{l+s}{\PYZdl{}} The sum of the vectors is commutative: \PY{l+s}{\PYZdl{}}\PY{n+nb}{a }\PY{o}{+}\PY{n+nb}{ b }\PY{o}{=}\PY{n+nb}{ b }\PY{o}{+}\PY{n+nb}{ a}\PY{l+s}{\PYZdl{}}.
          \PY{l+s+sb}{\PYZdl{}\PYZdl{}}\PY{l+s}{\PYZdl{}\PYZdl{}}
          \PY{l+s}{\PYZdl{}}\PY{n+nv}{\PYZbs{}mathbf}\PY{n+nb}{\PYZob{}}\PY{n+nb}{Property\PYZti{}}\PY{l+m}{2}\PY{n+nb}{:}\PY{n+nb}{\PYZcb{}}\PY{l+s}{\PYZdl{}} The sum of the vectors is associative: \PY{l+s}{\PYZdl{}}\PY{n+nb}{a }\PY{o}{+}\PY{n+nb}{ }\PY{o}{(}\PY{n+nb}{b }\PY{o}{+}\PY{n+nb}{ c}\PY{o}{)}\PY{n+nb}{ }\PY{o}{=}\PY{n+nb}{ }\PY{o}{(}\PY{n+nb}{a }\PY{o}{+}\PY{n+nb}{ b}\PY{o}{)}\PY{n+nb}{ }\PY{o}{+}\PY{n+nb}{ c }\PY{l+s}{\PYZdl{}}.
          \PY{l+s+sb}{\PYZdl{}\PYZdl{}}\PY{l+s}{\PYZdl{}\PYZdl{}}
          \PY{l+s}{\PYZdl{}}\PY{n+nv}{\PYZbs{}mathbf}\PY{n+nb}{\PYZob{}}\PY{n+nb}{Property\PYZti{}}\PY{l+m}{3}\PY{n+nb}{:}\PY{n+nb}{\PYZcb{}}\PY{l+s}{\PYZdl{}} Multiplication of a vector with a scalar is distributive with respect to the sum of
          vectors: \PY{l+s}{\PYZdl{}}\PY{n+nv}{\PYZbs{}alpha}\PY{o}{(}\PY{n+nb}{a }\PY{o}{+}\PY{n+nb}{ b}\PY{o}{)}\PY{n+nb}{ }\PY{o}{=}\PY{n+nb}{ }\PY{n+nv}{\PYZbs{}alpha}\PY{n+nb}{ a }\PY{o}{+}\PY{n+nb}{ }\PY{n+nv}{\PYZbs{}alpha}\PY{n+nb}{ b}\PY{l+s}{\PYZdl{}} and according to the sum of scalars: \PY{l+s}{\PYZdl{}}\PY{o}{(}\PY{n+nv}{\PYZbs{}alpha}\PY{n+nb}{ }\PY{o}{+}\PY{n+nb}{ }\PY{n+nv}{\PYZbs{}beta}\PY{o}{)}\PY{n+nb}{a }\PY{o}{=}\PY{n+nb}{ }\PY{n+nv}{\PYZbs{}alpha}\PY{n+nb}{ a }\PY{o}{+}\PY{n+nb}{ }\PY{n+nv}{\PYZbs{}beta}\PY{n+nb}{ a}\PY{l+s}{\PYZdl{}}
          \PY{l+s+sb}{\PYZdl{}\PYZdl{}}\PY{l+s}{\PYZdl{}\PYZdl{}}
          \PY{l+s}{\PYZdl{}}\PY{n+nv}{\PYZbs{}mathbf}\PY{n+nb}{\PYZob{}}\PY{n+nb}{Proof:}\PY{n+nb}{\PYZcb{}}\PY{l+s}{\PYZdl{}} All three properties are a simple consequence of the corresponding properties of real or complex numbers.
\end{Verbatim}


    The vector sum and product of a vector with a scalar have some similar properties
as the sum and product of the numbers.
$$$$
$\mathbf{Property~1:}$ The sum of the vectors is commutative: $a + b = b + a$.
$$$$
$\mathbf{Property~2:}$ The sum of the vectors is associative: $a + (b + c) = (a + b) + c $.
$$$$
$\mathbf{Property~3:}$ Multiplication of a vector with a scalar is distributive with respect to the sum of
vectors: $\alpha(a + b) = \alpha a + \alpha b$ and according to the sum of scalars: $(\alpha + \beta)a = \alpha a + \beta a$
$$$$
$\mathbf{Proof:}$ All three properties are a simple consequence of the corresponding properties of real or complex numbers.

    
    \hypertarget{linear-combination}{%
\subsubsection{Linear combination}\label{linear-combination}}

    \begin{Verbatim}[commandchars=\\\{\}]
{\color{incolor}In [{\color{incolor}118}]:} \PY{c}{\PYZpc{}\PYZpc{}latex}
          When a product with a scalar is combined with the assembly of vectors, we get to linear combinations of vectors.
          
          \PY{l+s+sb}{\PYZdl{}\PYZdl{}}\PY{l+s}{\PYZdl{}\PYZdl{}}
          \PY{l+s}{\PYZdl{}}\PY{n+nv}{\PYZbs{}mathbf}\PY{n+nb}{\PYZob{}}\PY{n+nb}{Definition:}\PY{n+nb}{\PYZcb{}}\PY{l+s}{\PYZdl{}} The linear combination of the vectors \PY{l+s}{\PYZdl{}}\PY{n+nb}{x}\PY{l+s}{\PYZdl{}} and \PY{l+s}{\PYZdl{}}\PY{n+nb}{y}\PY{l+s}{\PYZdl{}} is the sum \PY{l+s}{\PYZdl{}}\PY{n+nv}{\PYZbs{}alpha}\PY{n+nb}{ x }\PY{o}{+}\PY{n+nb}{ }\PY{n+nv}{\PYZbs{}beta}\PY{n+nb}{ y}\PY{l+s}{\PYZdl{}}.
          \PY{l+s+sb}{\PYZdl{}\PYZdl{}}\PY{l+s}{\PYZdl{}\PYZdl{}}
          \PY{l+s+sb}{\PYZdl{}\PYZdl{}}
          \PY{n+nb}{ }\PY{n+nv}{\PYZbs{}alpha}\PY{n+nb}{ x }\PY{o}{+}\PY{n+nb}{ }\PY{n+nv}{\PYZbs{}beta}\PY{n+nb}{ y }\PY{o}{=}\PY{n+nb}{ }\PY{n+nv}{\PYZbs{}alpha}\PY{n+nb}{ }\PY{n+nv}{\PYZbs{}begin}\PY{n+nb}{\PYZob{}}\PY{n+nb}{bmatrix}\PY{n+nb}{\PYZcb{}}\PY{n+nb}{x}\PY{n+nb}{\PYZus{}}\PY{l+m}{1}\PY{n+nb}{ }\PY{n+nv}{\PYZbs{}\PYZbs{}}\PY{n+nb}{ x}\PY{n+nb}{\PYZus{}}\PY{l+m}{2}\PY{n+nb}{ }\PY{n+nv}{\PYZbs{}\PYZbs{}}\PY{n+nb}{ ... }\PY{n+nv}{\PYZbs{}\PYZbs{}}\PY{n+nb}{ x}\PY{n+nb}{\PYZus{}}\PY{n+nb}{n}\PY{n+nv}{\PYZbs{}end}\PY{n+nb}{\PYZob{}}\PY{n+nb}{bmatrix}\PY{n+nb}{\PYZcb{}}\PY{n+nb}{ }\PY{o}{+}\PY{n+nb}{ }
          \PY{n+nb}{       }\PY{n+nv}{\PYZbs{}beta}\PY{n+nb}{ }\PY{n+nv}{\PYZbs{}begin}\PY{n+nb}{\PYZob{}}\PY{n+nb}{bmatrix}\PY{n+nb}{\PYZcb{}}\PY{n+nb}{y}\PY{n+nb}{\PYZus{}}\PY{l+m}{1}\PY{n+nb}{ }\PY{n+nv}{\PYZbs{}\PYZbs{}}\PY{n+nb}{ y}\PY{n+nb}{\PYZus{}}\PY{l+m}{2}\PY{n+nb}{ }\PY{n+nv}{\PYZbs{}\PYZbs{}}\PY{n+nb}{ ... }\PY{n+nv}{\PYZbs{}\PYZbs{}}\PY{n+nb}{ y}\PY{n+nb}{\PYZus{}}\PY{n+nb}{n}\PY{n+nv}{\PYZbs{}end}\PY{n+nb}{\PYZob{}}\PY{n+nb}{bmatrix}\PY{n+nb}{\PYZcb{}}
          \PY{l+s}{\PYZdl{}\PYZdl{}}
          \PY{l+s+sb}{\PYZdl{}\PYZdl{}}\PY{l+s}{\PYZdl{}\PYZdl{}}
          Similarly, a linear combination of several vectors can be assembled, e.g.
          \PY{l+s}{\PYZdl{}}\PY{n+nv}{\PYZbs{}alpha}\PY{n+nb}{ a }\PY{o}{+}\PY{n+nb}{ }\PY{n+nv}{\PYZbs{}beta}\PY{n+nb}{ b }\PY{o}{+}\PY{n+nb}{ · · · }\PY{o}{+}\PY{n+nb}{ }\PY{n+nv}{\PYZbs{}zeta}\PY{n+nb}{ z}\PY{l+s}{\PYZdl{}} is a linear combination of vectors \PY{l+s}{\PYZdl{}}\PY{n+nb}{a, b, ... , z}\PY{l+s}{\PYZdl{}}.
          
          \PY{l+s+sb}{\PYZdl{}\PYZdl{}}\PY{l+s}{\PYZdl{}\PYZdl{}}
          \PY{l+s}{\PYZdl{}}\PY{n+nv}{\PYZbs{}mathbf}\PY{n+nb}{\PYZob{}}\PY{n+nb}{Attention}\PY{o}{!}\PY{n+nb}{\PYZcb{}}\PY{l+s}{\PYZdl{}} All vectors in the liner combination must have the same number of components!
          \PY{l+s+sb}{\PYZdl{}\PYZdl{}}\PY{l+s}{\PYZdl{}\PYZdl{}}
          \PY{l+s}{\PYZdl{}}\PY{n+nv}{\PYZbs{}mathbf}\PY{n+nb}{\PYZob{}}\PY{n+nb}{Note:}\PY{n+nb}{\PYZcb{}}\PY{l+s}{\PYZdl{}} A set of all linear combinations of two vectors \PY{l+s}{\PYZdl{}}\PY{n+nb}{a}\PY{l+s}{\PYZdl{}} and \PY{l+s}{\PYZdl{}}\PY{n+nb}{b}\PY{l+s}{\PYZdl{}} is a plane, except
          when the vectors are collinear. In this case, there is a set of all linear combinations
          which is the line on which both vectors lie.
\end{Verbatim}


    When a product with a scalar is combined with the assembly of vectors, we get to linear combinations of vectors.

$$$$
$\mathbf{Definition:}$ The linear combination of the x and y vectors is the sum $\alpha x + \beta y$.
$$$$
$$
 \alpha x + \beta y = \alpha \begin{bmatrix}x_1 \\ x_2 \\ ... \\ x_n\end{bmatrix} + 
       \beta \begin{bmatrix}y_1 \\ y_2 \\ ... \\ y_n\end{bmatrix}
$$
$$$$
Similarly, a linear combination of several vectors can be assembled, e.g.
$\alpha a + \beta b + · · · + \zeta z$ is a linear combination of vectors $a, b, ... , z$.

$$$$
$\mathbf{Attention!}$ All vectors in the liner combination must have the same number of components!
$$$$
$\mathbf{Note:}$ A set of all linear combinations of two vectors $a$ and $b$ is a plane, except
when the vectors are collinear. In this case, there is a set of all linear combinations
which is the line on which both vectors lie.

    
    \hypertarget{scalar-product}{%
\subsubsection{Scalar product}\label{scalar-product}}

    \begin{Verbatim}[commandchars=\\\{\}]
{\color{incolor}In [{\color{incolor}146}]:} \PY{c}{\PYZpc{}\PYZpc{}latex}
          An important operation over two vectors is a scalar product.
          
          \PY{l+s+sb}{\PYZdl{}\PYZdl{}}\PY{l+s}{\PYZdl{}\PYZdl{}}
          \PY{l+s}{\PYZdl{}}\PY{n+nv}{\PYZbs{}mathbf}\PY{n+nb}{\PYZob{}}\PY{n+nb}{Definition:}\PY{n+nb}{\PYZcb{}}\PY{l+s}{\PYZdl{}} The scalar product of the vectors
          \PY{l+s}{\PYZdl{}}\PY{n+nb}{x }\PY{o}{=}\PY{n+nb}{ }\PY{n+nv}{\PYZbs{}begin}\PY{n+nb}{\PYZob{}}\PY{n+nb}{bmatrix}\PY{n+nb}{\PYZcb{}}\PY{n+nb}{x}\PY{n+nb}{\PYZus{}}\PY{l+m}{1}\PY{n+nb}{ }\PY{n+nv}{\PYZbs{}\PYZbs{}}\PY{n+nb}{ x}\PY{n+nb}{\PYZus{}}\PY{l+m}{2}\PY{n+nb}{ }\PY{n+nv}{\PYZbs{}\PYZbs{}}\PY{n+nb}{ ... }\PY{n+nv}{\PYZbs{}\PYZbs{}}\PY{n+nb}{ x}\PY{n+nb}{\PYZus{}}\PY{n+nb}{n}\PY{n+nv}{\PYZbs{}end}\PY{n+nb}{\PYZob{}}\PY{n+nb}{bmatrix}\PY{n+nb}{\PYZcb{}}\PY{l+s}{\PYZdl{}} and 
          \PY{l+s}{\PYZdl{}}\PY{n+nb}{y }\PY{o}{=}\PY{n+nb}{ }\PY{n+nv}{\PYZbs{}begin}\PY{n+nb}{\PYZob{}}\PY{n+nb}{bmatrix}\PY{n+nb}{\PYZcb{}}\PY{n+nb}{y}\PY{n+nb}{\PYZus{}}\PY{l+m}{1}\PY{n+nb}{ }\PY{n+nv}{\PYZbs{}\PYZbs{}}\PY{n+nb}{ y}\PY{n+nb}{\PYZus{}}\PY{l+m}{2}\PY{n+nb}{ }\PY{n+nv}{\PYZbs{}\PYZbs{}}\PY{n+nb}{ ... }\PY{n+nv}{\PYZbs{}\PYZbs{}}\PY{n+nb}{ y}\PY{n+nb}{\PYZus{}}\PY{n+nb}{n}\PY{n+nv}{\PYZbs{}end}\PY{n+nb}{\PYZob{}}\PY{n+nb}{bmatrix}\PY{n+nb}{\PYZcb{}}\PY{l+s}{\PYZdl{}} is the number \PY{l+s}{\PYZdl{}}\PY{n+nb}{x }\PY{n+nv}{\PYZbs{}cdot}\PY{n+nb}{ y }\PY{o}{=}\PY{n+nb}{ x}\PY{n+nb}{\PYZus{}}\PY{l+m}{1}\PY{n+nb}{y}\PY{n+nb}{\PYZus{}}\PY{l+m}{1}\PY{n+nb}{ }\PY{o}{+}\PY{n+nb}{ x}\PY{n+nb}{\PYZus{}}\PY{l+m}{2}\PY{n+nb}{y}\PY{n+nb}{\PYZus{}}\PY{l+m}{2}\PY{n+nb}{ }\PY{o}{+}\PY{n+nb}{ · · · }\PY{o}{+}\PY{n+nb}{ x}\PY{n+nb}{\PYZus{}}\PY{n+nb}{ny}\PY{n+nb}{\PYZus{}}\PY{n+nb}{n}\PY{l+s}{\PYZdl{}}.
          \PY{l+s+sb}{\PYZdl{}\PYZdl{}}\PY{l+s}{\PYZdl{}\PYZdl{}}
          
          \PY{l+s}{\PYZdl{}}\PY{n+nv}{\PYZbs{}mathbf}\PY{n+nb}{\PYZob{}}\PY{n+nb}{Attention}\PY{o}{!}\PY{n+nb}{\PYZcb{}}\PY{l+s}{\PYZdl{}}  Vectors used in scalar product multiplication must have the same number of components!
          
          \PY{l+s+sb}{\PYZdl{}\PYZdl{}}\PY{l+s}{\PYZdl{}\PYZdl{}}
          
          It can be easily verified that the scalar product has the following properties:
              
          \PY{l+s+sb}{\PYZdl{}\PYZdl{}}\PY{l+s}{\PYZdl{}\PYZdl{}}
          \PY{l+s}{\PYZdl{}}\PY{n+nv}{\PYZbs{}mathbf}\PY{n+nb}{\PYZob{}}\PY{n+nb}{Property\PYZti{}}\PY{l+m}{1}\PY{n+nb}{:}\PY{n+nb}{\PYZcb{}}\PY{l+s}{\PYZdl{}} Commutativity: \PY{l+s}{\PYZdl{}}\PY{n+nb}{x · y }\PY{o}{=}\PY{n+nb}{ y · x}\PY{l+s}{\PYZdl{}}.
          \PY{l+s+sb}{\PYZdl{}\PYZdl{}}\PY{l+s}{\PYZdl{}\PYZdl{}}
          \PY{l+s}{\PYZdl{}}\PY{n+nv}{\PYZbs{}mathbf}\PY{n+nb}{\PYZob{}}\PY{n+nb}{Property\PYZti{}}\PY{l+m}{2}\PY{n+nb}{:}\PY{n+nb}{\PYZcb{}}\PY{l+s}{\PYZdl{}} Aditivity: \PY{l+s}{\PYZdl{}}\PY{n+nb}{x · }\PY{o}{(}\PY{n+nb}{y }\PY{o}{+}\PY{n+nb}{ z}\PY{o}{)}\PY{n+nb}{ }\PY{o}{=}\PY{n+nb}{ x · y }\PY{o}{+}\PY{n+nb}{ x · z}\PY{l+s}{\PYZdl{}}.
          \PY{l+s+sb}{\PYZdl{}\PYZdl{}}\PY{l+s}{\PYZdl{}\PYZdl{}}
          \PY{l+s}{\PYZdl{}}\PY{n+nv}{\PYZbs{}mathbf}\PY{n+nb}{\PYZob{}}\PY{n+nb}{Property\PYZti{}}\PY{l+m}{3}\PY{n+nb}{:}\PY{n+nb}{\PYZcb{}}\PY{l+s}{\PYZdl{}} Homogeneity: \PY{l+s}{\PYZdl{}}\PY{n+nb}{x · }\PY{o}{(}\PY{n+nv}{\PYZbs{}alpha}\PY{n+nb}{ y}\PY{o}{)}\PY{n+nb}{ }\PY{o}{=}\PY{n+nb}{ }\PY{n+nv}{\PYZbs{}alpha}\PY{o}{(}\PY{n+nb}{x · y}\PY{o}{)}\PY{n+nb}{ }\PY{o}{=}\PY{n+nb}{ }\PY{o}{(}\PY{n+nv}{\PYZbs{}alpha}\PY{n+nb}{ x}\PY{o}{)}\PY{n+nb}{ · y}\PY{l+s}{\PYZdl{}}.
          \PY{l+s+sb}{\PYZdl{}\PYZdl{}}\PY{l+s}{\PYZdl{}\PYZdl{}}
          \PY{l+s}{\PYZdl{}}\PY{n+nv}{\PYZbs{}mathbf}\PY{n+nb}{\PYZob{}}\PY{n+nb}{Property\PYZti{}}\PY{l+m}{4}\PY{n+nb}{:}\PY{n+nb}{\PYZcb{}}\PY{l+s}{\PYZdl{}} Positive definiteness: for each vector \PY{l+s}{\PYZdl{}}\PY{n+nb}{x}\PY{l+s}{\PYZdl{}}, \PY{l+s}{\PYZdl{}}\PY{n+nb}{x }\PY{n+nv}{\PYZbs{}cdot}\PY{n+nb}{ x }\PY{n+nv}{\PYZbs{}ge}\PY{n+nb}{ }\PY{l+m}{0}\PY{l+s}{\PYZdl{}}. If \PY{l+s}{\PYZdl{}}\PY{n+nb}{x }\PY{n+nv}{\PYZbs{}cdot}\PY{n+nb}{ x }\PY{o}{=}\PY{n+nb}{ }\PY{l+m}{0}\PY{l+s}{\PYZdl{}}, then \PY{l+s}{\PYZdl{}}\PY{n+nb}{x }\PY{o}{=}\PY{n+nb}{ }\PY{l+m}{0}\PY{l+s}{\PYZdl{}}.
          \PY{l+s+sb}{\PYZdl{}\PYZdl{}}\PY{l+s}{\PYZdl{}\PYZdl{}}
          
          \PY{l+s}{\PYZdl{}}\PY{n+nv}{\PYZbs{}mathbf}\PY{n+nb}{\PYZob{}}\PY{n+nb}{Proof:}\PY{n+nb}{\PYZcb{}}\PY{l+s}{\PYZdl{}} TODO for all.
\end{Verbatim}


    An important operation over two vectors is a scalar product.

$$$$
$\mathbf{Definition:}$ The scalar product of the vectors
$x = \begin{bmatrix}x_1 \\ x_2 \\ ... \\ x_n\end{bmatrix}$ and 
$y = \begin{bmatrix}y_1 \\ y_2 \\ ... \\ y_n\end{bmatrix}$ is the number $x \cdot y = x_1y_1 + x_2y_2 + · · · + x_ny_n$.
$$$$

$\mathbf{Attention!}$  Vectors used in scalar product multiplication must have the same number of components!

$$$$

It can be easily verified that the scalar product has the following properties:
    
$$$$
$\mathbf{Property~1:}$ Commutativity: $x · y = y · x$.
$$$$
$\mathbf{Property~2:}$ Aditivity: $x · (y + z) = x · y + x · z$.
$$$$
$\mathbf{Property~3:}$ Homogeneity: $x · (\alpha y) = \alpha(x · y) = (\alpha x) · y$.
$$$$
$\mathbf{Property~4:}$ Positive definiteness: for each vector $x$, $x \cdot x \ge 0$. If $x \cdot x = 0$, then $x = 0$.
$$$$

$\mathbf{Proof:}$ TODO for all.

    
    \hypertarget{vector-length-magnitude}{%
\subsubsection{Vector length
(magnitude)}\label{vector-length-magnitude}}

    \begin{Verbatim}[commandchars=\\\{\}]
{\color{incolor}In [{\color{incolor}28}]:} \PY{c}{\PYZpc{}\PYZpc{}latex}
         In the set \PY{l+s}{\PYZdl{}}\PY{n+nv}{\PYZbs{}mathbb}\PY{n+nb}{\PYZob{}}\PY{n+nb}{R}\PY{n+nb}{\PYZcb{}}\PY{n+nb}{\PYZca{}}\PY{l+m}{2}\PY{l+s}{\PYZdl{}} we can easily follow Pitagoras rule to calculate the length of a vector. 
         The vector is formed together with the \PY{l+s}{\PYZdl{}}\PY{n+nb}{x}\PY{l+s}{\PYZdl{}}\PYZhy{}axis and the parallel to the \PY{l+s}{\PYZdl{}}\PY{n+nb}{y}\PY{l+s}{\PYZdl{}}\PYZhy{}axis and determines
         a rectangular triangle whose cathets are its \PY{l+s}{\PYZdl{}}\PY{n+nb}{x}\PY{l+s}{\PYZdl{}} and \PY{l+s}{\PYZdl{}}\PY{n+nb}{y}\PY{l+s}{\PYZdl{}} coordinates. 
         Length of the vector (hypotenuse of a rectangular triangle) is therefore \PY{l+s}{\PYZdl{}}\PY{n+nv}{\PYZbs{}sqrt}\PY{n+nb}{\PYZob{}}\PY{n+nb}{x}\PY{n+nb}{\PYZca{}}\PY{l+m}{2}\PY{n+nb}{ }\PY{o}{+}\PY{n+nb}{ y}\PY{n+nb}{\PYZca{}}\PY{l+m}{2}\PY{n+nb}{\PYZcb{}}\PY{l+s}{\PYZdl{}}.
         
         \PY{l+s+sb}{\PYZdl{}\PYZdl{}}\PY{l+s}{\PYZdl{}\PYZdl{}}
         For vectors from \PY{l+s}{\PYZdl{}}\PY{n+nv}{\PYZbs{}mathbb}\PY{n+nb}{\PYZob{}}\PY{n+nb}{R}\PY{n+nb}{\PYZcb{}}\PY{n+nb}{\PYZca{}}\PY{n+nb}{N}\PY{l+s}{\PYZdl{}} the length of the vector can be similarly calculated as
         \PY{l+s}{\PYZdl{}}\PY{n+nb}{x}\PY{n+nb}{\PYZus{}}\PY{l+m}{1}\PY{n+nb}{\PYZca{}}\PY{l+m}{2}\PY{n+nb}{ }\PY{o}{+}\PY{n+nb}{ x}\PY{n+nb}{\PYZus{}}\PY{l+m}{2}\PY{n+nb}{\PYZca{}}\PY{l+m}{2}\PY{n+nb}{ }\PY{o}{+}\PY{n+nb}{ ... }\PY{o}{+}\PY{n+nb}{ x}\PY{n+nb}{\PYZus{}}\PY{n+nb}{n}\PY{n+nb}{\PYZca{}}\PY{l+m}{2}\PY{l+s}{\PYZdl{}}, which can be written in a vector form as \PY{l+s}{\PYZdl{}}\PY{n+nv}{\PYZbs{}sqrt}\PY{n+nb}{\PYZob{}}\PY{n+nb}{x}\PY{n+nb}{\PYZca{}}\PY{l+m}{2}\PY{n+nb}{ }\PY{o}{+}\PY{n+nb}{ x}\PY{n+nb}{\PYZca{}}\PY{l+m}{2}\PY{n+nb}{\PYZcb{}}\PY{l+s}{\PYZdl{}}.
             
         
         
         \PY{l+s+sb}{\PYZdl{}\PYZdl{}}\PY{l+s}{\PYZdl{}\PYZdl{}}
         \PY{l+s}{\PYZdl{}}\PY{n+nv}{\PYZbs{}mathbf}\PY{n+nb}{\PYZob{}}\PY{n+nb}{Definition:}\PY{n+nb}{\PYZcb{}}\PY{l+s}{\PYZdl{}} The length of the vector x is
         \PY{l+s}{\PYZdl{}}\PY{n+nb}{ ||x|| }\PY{o}{=}\PY{n+nb}{ }\PY{n+nv}{\PYZbs{}sqrt}\PY{n+nb}{\PYZob{}}\PY{n+nb}{x}\PY{n+nb}{\PYZca{}}\PY{l+m}{2}\PY{n+nb}{ }\PY{o}{+}\PY{n+nb}{ x}\PY{n+nb}{\PYZca{}}\PY{l+m}{2}\PY{n+nb}{\PYZcb{}}\PY{l+s}{\PYZdl{}}.
         \PY{l+s+sb}{\PYZdl{}\PYZdl{}}\PY{l+s}{\PYZdl{}\PYZdl{}}
         
         \PY{l+s}{\PYZdl{}}\PY{n+nv}{\PYZbs{}mathbf}\PY{n+nb}{\PYZob{}}\PY{n+nb}{Example:}\PY{n+nb}{\PYZcb{}}\PY{l+s}{\PYZdl{}}  
         
         \PY{l+s+sb}{\PYZdl{}\PYZdl{}}\PY{l+s}{\PYZdl{}\PYZdl{}}
         
         Vectors with a length one have a special role among the vectors.
         
         \PY{l+s+sb}{\PYZdl{}\PYZdl{}}\PY{l+s}{\PYZdl{}\PYZdl{}}
         \PY{l+s}{\PYZdl{}}\PY{n+nv}{\PYZbs{}mathbf}\PY{n+nb}{\PYZob{}}\PY{n+nb}{Definition:}\PY{n+nb}{\PYZcb{}}\PY{l+s}{\PYZdl{}} The unit vector is a vector with a length of \PY{l+s}{\PYZdl{}}\PY{l+m}{1}\PY{l+s}{\PYZdl{}}.
         \PY{l+s+sb}{\PYZdl{}\PYZdl{}}\PY{l+s}{\PYZdl{}\PYZdl{}}
\end{Verbatim}


    In the set $\mathbb{R}^2$ we can easily follow Pitagoras rule to calculate the length of a vector. 
The vector is formed together with the $x$-axis and the parallel to the $y$-axis and determines
a rectangular triangle whose cathets are its $x$ and $y$ coordinates. 
Length of the vector (hypotenuse of a rectangular triangle) is therefore $\sqrt{x^2 + y^2}$.

$$$$
For vectors from $\mathbb{R}^N$ the length of the vector can be similarly calculated as
$x_1^2 + x_2^2 + ... + x_n^2$, which can be written in a vector form as $\sqrt{x^2 + x^2}$.
    


$$$$
$\mathbf{Definition:}$ The length of the vector x is
$ ||x|| = \sqrt{x^2 + x^2}$.
$$$$

$\mathbf{Example:}$  

$$$$

Vectors with a length one have a special role among the vectors.

$$$$
$\mathbf{Definition:}$ The unit vector is a vector with a length of $1$.
$$$$

    
    Vectors: magnitude, direction, norms, dot product, linearity (0.5 hours)

    \hypertarget{vector-direction}{%
\subsubsection{Vector direction}\label{vector-direction}}

    \hypertarget{angle-between-two-vectors}{%
\subsubsection{Angle between two
vectors}\label{angle-between-two-vectors}}

    \hypertarget{vector-product}{%
\subsubsection{Vector product}\label{vector-product}}

    \begin{Verbatim}[commandchars=\\\{\}]
{\color{incolor}In [{\color{incolor}30}]:} \PY{k+kn}{import} \PY{n+nn}{numpy} \PY{k}{as} \PY{n+nn}{np}
         \PY{k+kn}{import} \PY{n+nn}{matplotlib} \PY{k}{as} \PY{n+nn}{plt}
\end{Verbatim}


    Operations: sum, sub, mul, div, norm

    Magnitude

    \begin{Verbatim}[commandchars=\\\{\}]
{\color{incolor}In [{\color{incolor}15}]:} \PY{n}{a} \PY{o}{=} \PY{n}{np}\PY{o}{.}\PY{n}{array}\PY{p}{(}\PY{p}{[}\PY{l+m+mi}{0}\PY{p}{,} \PY{l+m+mi}{1}\PY{p}{,} \PY{l+m+mi}{2}\PY{p}{]}\PY{p}{)}
         \PY{n}{b} \PY{o}{=} \PY{n}{np}\PY{o}{.}\PY{n}{array}\PY{p}{(}\PY{p}{[}\PY{o}{\PYZhy{}}\PY{l+m+mi}{1}\PY{p}{,}\PY{o}{\PYZhy{}}\PY{l+m+mi}{2}\PY{p}{,}\PY{o}{\PYZhy{}}\PY{l+m+mi}{1}\PY{p}{]}\PY{p}{)}
\end{Verbatim}


    \begin{Verbatim}[commandchars=\\\{\}]
{\color{incolor}In [{\color{incolor}18}]:} \PY{k+kn}{import} \PY{n+nn}{math}
         \PY{c+c1}{\PYZsh{} manually defining magnitude function}
         \PY{k}{def} \PY{n+nf}{mag}\PY{p}{(}\PY{n}{x}\PY{p}{)}\PY{p}{:} 
             \PY{k}{return} \PY{n}{math}\PY{o}{.}\PY{n}{sqrt}\PY{p}{(}\PY{n+nb}{sum}\PY{p}{(}\PY{n}{i}\PY{o}{*}\PY{o}{*}\PY{l+m+mi}{2} \PY{k}{for} \PY{n}{i} \PY{o+ow}{in} \PY{n}{x}\PY{p}{)}\PY{p}{)}
         
         \PY{n}{mag}\PY{p}{(}\PY{n}{a}\PY{p}{)}
\end{Verbatim}


\begin{Verbatim}[commandchars=\\\{\}]
{\color{outcolor}Out[{\color{outcolor}18}]:} 2.23606797749979
\end{Verbatim}
            
    \begin{Verbatim}[commandchars=\\\{\}]
{\color{incolor}In [{\color{incolor}19}]:} \PY{c+c1}{\PYZsh{} import from numpy}
         \PY{n}{np}\PY{o}{.}\PY{n}{linalg}\PY{o}{.}\PY{n}{norm}\PY{p}{(}\PY{n}{a}\PY{p}{,}\PY{n+nb}{ord}\PY{o}{=}\PY{l+m+mi}{2}\PY{p}{)}
\end{Verbatim}


\begin{Verbatim}[commandchars=\\\{\}]
{\color{outcolor}Out[{\color{outcolor}19}]:} 2.2360679774997898
\end{Verbatim}
            
    \begin{Verbatim}[commandchars=\\\{\}]
{\color{incolor}In [{\color{incolor}51}]:} \PY{k+kn}{import} \PY{n+nn}{matplotlib}\PY{n+nn}{.}\PY{n+nn}{pyplot} \PY{k}{as} \PY{n+nn}{plt}
         
         \PY{n}{lim} \PY{o}{=} \PY{l+m+mi}{2}
         
         \PY{n}{plt}\PY{o}{.}\PY{n}{quiver}\PY{p}{(}\PY{p}{[}\PY{l+m+mi}{0}\PY{p}{,} \PY{l+m+mi}{0}\PY{p}{]}\PY{p}{,} \PY{p}{[}\PY{l+m+mi}{0}\PY{p}{,} \PY{l+m+mi}{0}\PY{p}{]}\PY{p}{,} \PY{p}{[}\PY{n}{a}\PY{p}{[}\PY{l+m+mi}{0}\PY{p}{]}\PY{p}{,} \PY{n}{b}\PY{p}{[}\PY{l+m+mi}{0}\PY{p}{]}\PY{p}{]}\PY{p}{,} \PY{p}{[}\PY{n}{a}\PY{p}{[}\PY{l+m+mi}{1}\PY{p}{]}\PY{p}{,} \PY{n}{b}\PY{p}{[}\PY{l+m+mi}{1}\PY{p}{]}\PY{p}{]}\PY{p}{,} \PY{p}{[}\PY{n}{a}\PY{p}{[}\PY{l+m+mi}{2}\PY{p}{]}\PY{p}{,} \PY{n}{b}\PY{p}{[}\PY{l+m+mi}{2}\PY{p}{]}\PY{p}{]}\PY{p}{,} \PY{n}{color}\PY{o}{=}\PY{l+s+s1}{\PYZsq{}}\PY{l+s+s1}{rbg}\PY{l+s+s1}{\PYZsq{}}\PY{p}{,} \PY{n}{angles}\PY{o}{=}\PY{l+s+s1}{\PYZsq{}}\PY{l+s+s1}{xy}\PY{l+s+s1}{\PYZsq{}}\PY{p}{,} \PY{n}{scale\PYZus{}units}\PY{o}{=}\PY{l+s+s1}{\PYZsq{}}\PY{l+s+s1}{xy}\PY{l+s+s1}{\PYZsq{}}\PY{p}{,} \PY{n}{scale}\PY{o}{=}\PY{l+m+mi}{1}\PY{p}{)}
         \PY{n}{plt}\PY{o}{.}\PY{n}{xlim}\PY{p}{(}\PY{o}{\PYZhy{}}\PY{n}{lim}\PY{p}{,} \PY{n}{lim}\PY{p}{)}
         \PY{n}{plt}\PY{o}{.}\PY{n}{ylim}\PY{p}{(}\PY{o}{\PYZhy{}}\PY{n}{lim}\PY{p}{,} \PY{n}{lim}\PY{p}{)}
         \PY{n}{plt}\PY{o}{.}\PY{n}{show}\PY{p}{(}\PY{p}{)}
\end{Verbatim}


    \begin{center}
    \adjustimage{max size={0.9\linewidth}{0.9\paperheight}}{output_28_0.png}
    \end{center}
    { \hspace*{\fill} \\}
    
    \begin{Verbatim}[commandchars=\\\{\}]
{\color{incolor}In [{\color{incolor}23}]:} \PY{k+kn}{import} \PY{n+nn}{tensorflow} \PY{k}{as} \PY{n+nn}{tf}
\end{Verbatim}


    \begin{Verbatim}[commandchars=\\\{\}]
{\color{incolor}In [{\color{incolor}24}]:} \PY{c}{\PYZpc{}\PYZpc{}latex}
         
         \PY{l+s}{\PYZdl{}}\PY{n+nb}{a }\PY{o}{+}\PY{n+nb}{ b }\PY{o}{=}\PY{n+nb}{ c}\PY{l+s}{\PYZdl{}}
\end{Verbatim}


    
$a + b = c$

    
    \begin{Verbatim}[commandchars=\\\{\}]
{\color{incolor}In [{\color{incolor}25}]:} \PY{k+kn}{import} \PY{n+nn}{keras}
\end{Verbatim}


    \begin{Verbatim}[commandchars=\\\{\}]
Using TensorFlow backend.

    \end{Verbatim}

    \begin{Verbatim}[commandchars=\\\{\}]
{\color{incolor}In [{\color{incolor} }]:} 
\end{Verbatim}



    % Add a bibliography block to the postdoc
    
    
    
    \end{document}
